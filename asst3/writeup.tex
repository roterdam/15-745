\documentclass[11pt]{article}
\usepackage{enumerate}
\usepackage{tikz}
\usepackage{fullpage}
\usepackage{fancyhdr}
\usepackage{graphicx}

\usepackage{amsmath, amsfonts, amsthm, amssymb}
\setlength{\parindent}{0pt}
\setlength{\parskip}{5pt plus 1pt}
\pagestyle{empty}

\def\indented#1{\list{}{}\item[]}
\let\indented=\endlist

\newcounter{questionCounter}
\newcounter{partCounter}[questionCounter]
\newenvironment{question}[2][\arabic{questionCounter}]{%
    \setcounter{partCounter}{0}%
    \vspace{.25in} \hrule \vspace{0.5em}%
        \noindent{\bf #2}%
    \vspace{0.8em} \hrule \vspace{.10in}%
    \addtocounter{questionCounter}{1}%
}{}
\renewenvironment{part}[1][\alph{partCounter}]{%
    \addtocounter{partCounter}{1}%
    \vspace{.10in}%
    \begin{indented}%
       {\bf (#1)} %
}{\end{indented}}

%%%%%%%%%%%%%%%%% Identifying Information %%%%%%%%%%%%%%%%%
%% This is here, so that you can make your homework look %%
%% pretty when you compile it.                           %%
%%     DO NOT PUT YOUR NAME ANYWHERE ELSE!!!!            %%
%%%%%%%%%%%%%%%%%%%%%%%%%%%%%%%%%%%%%%%%%%%%%%%%%%%%%%%%%%%
\newcommand{\myname}{Michael Choquette, Rokhini Prabhu}
\newcommand{\myandrew}{mchoquet, rokhinip}
\newcommand{\myhwname}{Assignment 3}
%%%%%%%%%%%%%%%%%%%%%%%%%%%%%%%%%%%%%%%%%%%%%%%%%%%%%%%%%%%

\newcommand{\code}[1]{\texttt{#1}}

\begin{document}
\thispagestyle{plain}

\begin{center}
{\Large \myhwname} \\
\myname \\
\myandrew \\
\today
\end{center}

\begin{question}{Implementation Notes}

The loop-invariant code motion transformation works according to the following steps:
\begin{enumerate}
\item	Compute the nearest dominating ancestor of every block, using two instances of DFS.
\item	Extract all candidate instructions from the loop, where a candidate instruction is all of the following:
	\begin{itemize}
	\item Safe to speculatively execute.
	\item Not a memory read.
	\item	Not a phi node.
	\item	Not a landing pad instruction.
	\end{itemize}
\item Repeatedly loop through the list of candidates, extracting everything that is either not defined in the loop or only has loop-invariant operands. Stop when a pass through the candidates list generates no new instructions.
\item Loop through the generated list, and move all of them to the loop preheater.
\end{enumerate}
Some important notes:
\begin{itemize}
\item	As per the piazza post, we do not guarantee to only move instructions that dominate all loop exits.
\item	Because of the order in which we examined candidates, the instructions do not need to be topologically sorted before moving them to the preheater.
\end{itemize}
We evaluated performance by using the llvm interpreter instruction counter, as specified in the handout.


\end{question}

\newpage
\begin{question}{Register Allocation}

\end{question}
\newpage

\begin{question}{Instruction Scheduling}

The answers to parts 1 and 2 are in the following table:

\begin{tabular}{|c|c|c|c|c|}
\hline
Clock Cycle	&	Ready(forwards)	&	Issued(forwards)	&	Ready(backwards)	&	Issued(backwards)\\
\hline
0			&	(1, 2, 4)			&	(1)				&	(1, 2, 4)			&	(1)				\\
\hline
1			&	(2, 4)				&	(2)				&	(2, 4)				&	(2)				\\
\hline
2			&	(4)				&	(4)				&	(4)				&	()				\\
\hline
3			&	()				&	()				&	(4)				&	(4)				\\
\hline
4			&	(3, 5)				&	(3, 5)				&	(3, 5)				&	(5)				\\
\hline
5			&	(6, 7, 9, 10)		&	(6, 7)				&	(3, 7, 9)			&	(7)				\\
\hline
6			&	(8, 9, 10)			&	(9, 10)			&	(3, 6, 8, 9, 10)		&	(6, 10)			\\
\hline
7			&	(8)				&	(8)				&	(3, 8, 9)			&	(8, 9)				\\
\hline
8			&	(12)				&	(12)				&	(3)				&	(3)				\\
\hline
9			&	()				&	()				&	(12)				&	()				\\
\hline
10			&	(11, 13)			&	(11, 13)			&	(11, 12)			&	(12)				\\
\hline
11			&	(14)				&	(14)				&	(11, 13)			&	(11, 13)			\\
\hline
12			&	(15)				&	(15)				&	(14)				&	(14)				\\
\hline
13			&					&					&	(15)				&	(15)				\\
\hline
\end{tabular}

\end{question}
\end{document}
