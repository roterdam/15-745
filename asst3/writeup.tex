\documentclass[11pt]{article}
\usepackage{enumerate}
\usepackage{tikz}
\usepackage{fullpage}
\usepackage{fancyhdr}
\usepackage{graphicx}

\usepackage{amsmath, amsfonts, amsthm, amssymb}
\setlength{\parindent}{0pt}
\setlength{\parskip}{5pt plus 1pt}
\pagestyle{empty}

\def\indented#1{\list{}{}\item[]}
\let\indented=\endlist

\newcounter{questionCounter}
\newcounter{partCounter}[questionCounter]
\newenvironment{question}[2][\arabic{questionCounter}]{%
    \setcounter{partCounter}{0}%
    \vspace{.25in} \hrule \vspace{0.5em}%
        \noindent{\bf #2}%
    \vspace{0.8em} \hrule \vspace{.10in}%
    \addtocounter{questionCounter}{1}%
}{}
\renewenvironment{part}[1][\alph{partCounter}]{%
    \addtocounter{partCounter}{1}%
    \vspace{.10in}%
    \begin{indented}%
       {\bf (#1)} %
}{\end{indented}}

%%%%%%%%%%%%%%%%% Identifying Information %%%%%%%%%%%%%%%%%
%% This is here, so that you can make your homework look %%
%% pretty when you compile it.                           %%
%%     DO NOT PUT YOUR NAME ANYWHERE ELSE!!!!            %%
%%%%%%%%%%%%%%%%%%%%%%%%%%%%%%%%%%%%%%%%%%%%%%%%%%%%%%%%%%%
\newcommand{\myname}{Michael Choquette, Rokhini Prabhu}
\newcommand{\myandrew}{mchoquet, rokhinip}
\newcommand{\myhwname}{Assignment 3}
%%%%%%%%%%%%%%%%%%%%%%%%%%%%%%%%%%%%%%%%%%%%%%%%%%%%%%%%%%%

\newcommand{\code}[1]{\texttt{#1}}

\begin{document}
\thispagestyle{plain}

\begin{center}
{\Large \myhwname} \\
\myname \\
\myandrew \\
\today
\end{center}

\begin{question}{Implementation Notes}

\end{question}

\newpage
\begin{question}{Register Allocation}

\end{question}
\newpage

\begin{question}{Instruction Scheduling}

The answers to parts 1 and 2 are in the following table:

\begin{tabular}{|c|c|c|c|c|}
\hline
Clock Cycle	&	Ready(forwards)	&	Issued(forwards)	&	Ready(backwards)	&	Issued(backwards)\\
\hline
0			&	(1, 2, 4)			&	(1)				&	(1, 2, 4)			&	(1)				\\
\hline
1			&	(2, 4)				&	(2)				&	(2, 4)				&	(2)				\\
\hline
2			&	(4)				&	(4)				&	(4)				&	()				\\
\hline
3			&	()				&	()				&	(4)				&	(4)				\\
\hline
4			&	(3, 5)				&	(3, 5)				&	(3, 5)				&	(5)				\\
\hline
5			&	(6, 7, 9, 10)		&	(6, 7)				&	(3, 7, 9)			&	(7)				\\
\hline
6			&	(8, 9, 10)			&	(9, 10)			&	(3, 6, 8, 9, 10)		&	(6, 10)			\\
\hline
7			&	(8)				&	(8)				&	(3, 8, 9)			&	(8, 9)				\\
\hline
8			&	(12)				&	(12)				&	(3)				&	(3)				\\
\hline
9			&	()				&	()				&	(12)				&	()				\\
\hline
10			&	(11, 13)			&	(11, 13)			&	(11, 12)			&	(12)				\\
\hline
11			&	(14)				&	(14)				&	(11, 13)			&	(11, 13)			\\
\hline
12			&	(15)				&	(15)				&	(14)				&	(14)				\\
\hline
13			&					&					&	(15)				&	(15)				\\
\hline
\end{tabular}

\end{question}
\end{document}
