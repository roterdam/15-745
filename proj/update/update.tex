\documentclass[11pt]{article}
\usepackage{enumerate}
\usepackage{proof}
\usepackage{latexsym}
\usepackage{hyperref}
\usepackage{tikz}
\usepackage{fullpage}
\usepackage{fancyhdr}
\usepackage{graphicx}

\usepackage{amsmath, amsfonts, amsthm, amssymb}
\setlength{\parindent}{0pt}
\setlength{\parskip}{5pt plus 1pt}
\pagestyle{empty}

\def\indented#1{\list{}{}\item[]}
\let\indented=\endlist

\newcounter{questionCounter}
\newcounter{partCounter}[questionCounter]
\newenvironment{question}[2][\arabic{questionCounter}]{%
    \setcounter{partCounter}{0}%
    \vspace{.25in} \hrule \vspace{0.5em}%
        \noindent{\bf #2}%
    \vspace{0.8em} \hrule \vspace{.10in}%
    \addtocounter{questionCounter}{1}%
}{}
\renewenvironment{part}[1][\alph{partCounter}]{%
    \addtocounter{partCounter}{1}%
    \vspace{.10in}%
    \begin{indented}%
       {\bf (#1)} %
}{\end{indented}}

%%%%%%%%%%%%%%%%% Identifying Information %%%%%%%%%%%%%%%%%
%% This is here, so that you can make your homework look %%
%% pretty when you compile it.                           %%
%%     DO NOT PUT YOUR NAME ANYWHERE ELSE!!!!            %%
%%%%%%%%%%%%%%%%%%%%%%%%%%%%%%%%%%%%%%%%%%%%%%%%%%%%%%%%%%%
\newcommand{\myname}{Michael Choquette, Rokhini Prabhu}
\newcommand{\myandrew}{mchoquet, rokhinip}
\newcommand{\projectname}{Proposal Update} 
%%%%%%%%%%%%%%%%%%%%%%%%%%%%%%%%%%%%%%%%%%%%%%%%%%%%%%%%%%%

\begin{document}
\thispagestyle{plain}

\begin{center}
{\Large \projectname} \\
\myname \\
\myandrew \\
\today
\end{center}

\newcommand{\nonterm}[1]{$\langle${#1}$\rangle$}
\newcommand{\OR}{\ensuremath{\ | \ \ }}
\newcommand{\term}[1]{\textbf{#1}}
\newcommand{\code}[1]{\texttt{#1}}

\begin{question}{Major Changes}
There have been no major changes in our goals since the project proposal. 
We have however realized that unlike in our proposal, we couldn't
simply address the problem incrementally in three parts - coalescing 
\code{map} operations, coalescing \code{map} and \code{reduce} operations and 
coalescing \code{map} and \code{filter} operations - like we originally had 
planned. Instead, we realized that we needed to consider all the operations 
together while coming up with an algorithm to be able to coalesce operations 
together. Therefore, our original plan has been modified slightly.
\end{question}
\begin{question}{What you have accomplished so far}
We have formalized our implementation plans for this project by clearly
defining the grammar for the updated C0 language that we are designing. We have
also written up typing rules and have come up with an initial dataflow pass to
determine how far to delay executing \code{map}, \code{filter} and \code{reduce}
operations - as a first step to being able to then coalesce operations together.
We also currently have a fully functioning \code{C1} to C compiler which we have
augmented to handle the additional constructs necessary for sequences. 
\end{question}

\begin{question}{Meeting your milestone}
We have fallen behind on our goal of being able to coaslece consecutive
\code{map} operations. This has been because of two reasons - firstly we 
restructured our implementation plans and seconldy because we have not 
been able to put in as much time as we wanted to into this project due to 
other unexpectedly pressing matters which came up.
\end{question}

\begin{question}{Surprises}
There have been no major surprises in our tackling of the problem. 
\end{question}

\begin{question}{Revised Schedule}
\begin{center}
\begin{tabular}{|p{8cm}|c|c|}
\hline 
Work to do & Deadline & Who \\ \hline
Write a pass to determine how far down \code{map} and \code{filter} operations
can be pushed down in the code path without modifying loops and conditionals &
May 22nd & Rokhini \\ \hline
Based on the pass above, modify the compiler to produce valid C code that has 
lazy sequence evaluation & May 25th & Michael \\ \hline 
Write a pass for code motion into loops and conditionals when valid & May 29th &
Michael and Rokhini \\ \hline
Write the final report & May 2nd & Michael and Rokhini \\ \hline
\end{tabular}
\end{center}
\end{question}

\begin{question}{Resources Needed}
Yes we do. We don't require any other materials. 
\end{question}
\end{document}
