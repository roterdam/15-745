\documentclass[11pt]{article}
\usepackage{enumerate}
\usepackage{hyperref}
\usepackage{tikz}
\usepackage{fullpage}
\usepackage{fancyhdr}
\usepackage{graphicx}

\usepackage{amsmath, amsfonts, amsthm, amssymb}
\setlength{\parindent}{0pt}
\setlength{\parskip}{5pt plus 1pt}
\pagestyle{empty}

\def\indented#1{\list{}{}\item[]}
\let\indented=\endlist

\newcounter{questionCounter}
\newcounter{partCounter}[questionCounter]
\newenvironment{question}[2][\arabic{questionCounter}]{%
    \setcounter{partCounter}{0}%
    \vspace{.25in} \hrule \vspace{0.5em}%
        \noindent{\bf #2}%
    \vspace{0.8em} \hrule \vspace{.10in}%
    \addtocounter{questionCounter}{1}%
}{}
\renewenvironment{part}[1][\alph{partCounter}]{%
    \addtocounter{partCounter}{1}%
    \vspace{.10in}%
    \begin{indented}%
       {\bf (#1)} %
}{\end{indented}}

%%%%%%%%%%%%%%%%% Identifying Information %%%%%%%%%%%%%%%%%
%% This is here, so that you can make your homework look %%
%% pretty when you compile it.                           %%
%%     DO NOT PUT YOUR NAME ANYWHERE ELSE!!!!            %%
%%%%%%%%%%%%%%%%%%%%%%%%%%%%%%%%%%%%%%%%%%%%%%%%%%%%%%%%%%%
\newcommand{\myname}{Michael Choquette, Rokhini Prabhu}
\newcommand{\myandrew}{mchoquet, rokhinip}
\newcommand{\projectname}{Improving cache locality with coalesced map-reduce
operations} 
%%%%%%%%%%%%%%%%%%%%%%%%%%%%%%%%%%%%%%%%%%%%%%%%%%%%%%%%%%%

\newcommand{\code}[1]{\texttt{#1}}

\begin{document}
\thispagestyle{plain}

\begin{center}
{\Large \projectname} \\
\myname \\
\myandrew \\
\today
\end{center}

\begin{question}{Project Web Page}

\end{question}

\begin{question}{Project Description}
\texttt{C0} is a safe subset of the C programming language augmented with
contracts which has been developed here at CMU by Frank Pfenning and his team. 
In our project, we would like to extend the \texttt{C0} language with additional
language constructs for \texttt{map}, \texttt{reduce} and \texttt{filter}
constructs on immutable \texttt{sequences}. We chose \texttt{C0} since we have
both written full fledged compilers targetting \texttt{C0} and it is also a
small enough language for us to add additional constructs to. \\
\\ On an optimization level, we are interested in the speedups that can be
gained when we are performing several consecutive \texttt{map, reduce} 
and \texttt{filter} operations on large \texttt{sequences}. 
During such operations, we can leverage on speedups presented by
locality  by composing the functions of these constructs
together and applying them onto the sequence once instead of performing the
operations one at a time. This exposes the speedups that is gained from
locality. 
\\
\\ 
We will evaluate the impact of our optimizations by timing the performance of
benchmark tests. We expect our project to consist of 4 phases:
\begin{enumerate}
\item Integrating \texttt{map, reduce} and \texttt{filter} constructs into
\texttt{C0}
\item Coalesce multiple \texttt{map} operations into a single \texttt{map}
operation
\item Coalesce \texttt{map} and \texttt{reduce} operations together
\item Coalesce \texttt{map} and \texttt{filter} operations together
\end{enumerate}

As such, we would have achieved our goal to a 100\% if we are able to accomplish
all 4 stages of optimization. We also have a 75\% mark of having accomplished
only 3 out of the 4 stages. Our current optimization pass act on each function
individually and given extra time, we would also like to explore how such
optimizations would work on an inter-procedural level.
\end{question}
\newpage

\begin{question}{Plan of Attack and Schedule}
\begin{center}
\begin{tabular}{|p{5cm}|p{5cm}|p{5cm}|}
\hline
Week & Work to be done & Who \\ \hline
1 & Get C0 starter code, design grammar and typing rules for
\texttt{map, reduce} and \texttt{filter} constructs & Rokhini and Michael \\ \hline
2 & Implement \texttt{map, reduce} and \texttt{filter} in C0. Design rules for
dependency analysis for coalescing map functions & Rokhini and
Michael \\ \hline
3 & Implement rules for phase (2) of project, design rules for phase (3) and (4) &
Rokhini and Michael \\ \hline
4 and 5 & Implement rules for phase (3) and (4) of project & Rokhini and Michael \\ \hline
6 & Finish up any leftover work from week 4 and 5 and prepare for presentation &
Rokhini and Michael \\ \hline
\end{tabular}
\end{center}
\end{question}

\begin{question}{Milestone}
We can see that the milestone that we would like to have reached at the end of
week 3 is to have finished implementing composition of map functions and
determined how we are to implement phase (3) and (4) of the project.  
\end{question}

\begin{question}{Literature Search}
Our main reference for literature is the documentation for the
\href{http://www.cs.cmu.edu/~15210/docs/sig/SEQUENCE.html}{\textit{210 library on
Sequences}} for we plan to implement similar functionalities into \texttt{C0}. We
will also be using the
\href{http://c0.typesafety.net/doc/c0-reference.pdf}{\textit{Language guide for
C0}} to ensure that we still adhere to the safety features of the language. 
\end{question}
\begin{question}{Resources Needed}
Apart from the source code for a C0 compiler, we will not needing any special
resources or hardware for our project. We plan to run the code on the unix
machines to test the performance improvements. For the C0 compiler which we 
will be extending, we will be either using one of our own compilers from 
15-411 or the C0 reference compiler.
\end{question}
\begin{question}{Getting started}
We have not done much apart from writing the proposal so far. We do not foresee
any trouble with starting our project immediately
\end{question}

\end{document}
