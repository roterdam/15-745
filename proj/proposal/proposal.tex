\documentclass[11pt]{article}
\usepackage{enumerate}
\usepackage{hyperref}
\usepackage{tikz}
\usepackage{fullpage}
\usepackage{fancyhdr}
\usepackage{graphicx}

\usepackage{amsmath, amsfonts, amsthm, amssymb}
\setlength{\parindent}{0pt}
\setlength{\parskip}{5pt plus 1pt}
\pagestyle{empty}

\def\indented#1{\list{}{}\item[]}
\let\indented=\endlist

\newcounter{questionCounter}
\newcounter{partCounter}[questionCounter]
\newenvironment{question}[2][\arabic{questionCounter}]{%
    \setcounter{partCounter}{0}%
    \vspace{.25in} \hrule \vspace{0.5em}%
        \noindent{\bf #2}%
    \vspace{0.8em} \hrule \vspace{.10in}%
    \addtocounter{questionCounter}{1}%
}{}
\renewenvironment{part}[1][\alph{partCounter}]{%
    \addtocounter{partCounter}{1}%
    \vspace{.10in}%
    \begin{indented}%
       {\bf (#1)} %
}{\end{indented}}

%%%%%%%%%%%%%%%%% Identifying Information %%%%%%%%%%%%%%%%%
%% This is here, so that you can make your homework look %%
%% pretty when you compile it.                           %%
%%     DO NOT PUT YOUR NAME ANYWHERE ELSE!!!!            %%
%%%%%%%%%%%%%%%%%%%%%%%%%%%%%%%%%%%%%%%%%%%%%%%%%%%%%%%%%%%
\newcommand{\myname}{Michael Choquette, Rokhini Prabhu}
\newcommand{\myandrew}{mchoquet, rokhinip}
\newcommand{\projectname}{Improving Cache Locality with Coalesced Map-Reduce
Operations} 
%%%%%%%%%%%%%%%%%%%%%%%%%%%%%%%%%%%%%%%%%%%%%%%%%%%%%%%%%%%

\newcommand{\code}[1]{\texttt{#1}}

\begin{document}
\thispagestyle{plain}

\begin{center}
{\Large \projectname} \\
\myname \\
\myandrew \\
\today
\end{center}

\begin{question}{Project Web Page}
http://www.andrew.cmu.edu/user/mchoquet/15745/index.html
\end{question}

\begin{question}{Project Description}
\texttt{C0} is a safe subset of the C programming language augmented with
contracts which has been developed here at CMU by Frank Pfenning and his team. 
In our project, we will extend the \texttt{C0} language with additional
language constructs for \texttt{map}, \texttt{reduce} and \texttt{filter}
constructs on immutable \texttt{sequences}. We chose \texttt{C0} since we are both
comfortable with the language, and have access to a working \texttt{C0-C} compiler to
build off of.\\

Our goal is to investigate the speedups that can be gained from statically combining
sequential \texttt{map}, \texttt{reduce}, and \texttt{filter} operations into compound
operations by composing the kernels. This reduces the number of iterations through
the data, improving locality, and removes intermediate allocations.\\

We will evaluate the impact of our optimizations by timing the performance of
benchmark tests. We expect our project to consist of 4 phases:
\begin{enumerate}
\item Integrating \texttt{map, reduce} and \texttt{filter} constructs into
\texttt{C0}
\item Coalesce multiple \texttt{map} operations into a single \texttt{map}
operation
\item Coalesce \texttt{map} and \texttt{reduce} operations together
\item Coalesce \texttt{map} and \texttt{filter} operations together
\end{enumerate}

As such, we would have achieved our goal to a 100\% if we are able to accomplish
all 4 stages of optimization. We also have a 75\% mark of having accomplished
only 3 out of the 4 stages. Our current optimization pass act on each function
individually and given extra time, we would also like to explore how such
optimizations would work on an inter-procedural level.
\end{question}

\begin{question}{Literature Search}
Our main reference for literature is the documentation for the
\href{http://www.cs.cmu.edu/~15210/docs/sig/SEQUENCE.html}{\textit{210 library on
Sequences}} for we plan to implement similar functionalities into \texttt{C0}. We
will also be using the
\href{http://c0.typesafety.net/doc/c0-reference.pdf}{\textit{Language guide for
C0}} to ensure that we still adhere to the safety features of the language. 
\end{question}

\newpage

\begin{question}{Plan of Attack and Schedule}
\begin{center}
\begin{tabular}{|p{5cm}|p{5cm}|p{5cm}|}
\hline
Week & Work to be done & Who \\ \hline
1 & Get C0 starter code, design grammar and typing rules for
\texttt{map, reduce} and \texttt{filter} constructs. & Rokhini and Michael \\ \hline
2 & Implement \texttt{map, reduce} and \texttt{filter} in \texttt{C0}. Determine what
dependency analysis needs to be done for coalescing map functions, and how
to do it. &  Rokhini and Michael \\ \hline
3 & Implement rules for phase (2) of project, design rules for phase (3) and (4). &
Rokhini and Michael \\ \hline
4 and 5 & Implement rules for phase (3) and (4) of project. & Rokhini and Michael \\ \hline
6 & Finish up any leftover work from week 4 and 5 and prepare for presentation. &
Rokhini and Michael \\ \hline
\end{tabular}
\end{center}
\end{question}

\begin{question}{Milestone}
At the end of week 3 we would like to have finished implementing composition of
map operations, and have a strong understanding of how to combine reduce and filter
operations.
\end{question}

\begin{question}{Resources Needed}
We will use a \texttt{C0}-to-\texttt{C} compiler, but no other special resources. We
will run our code on the campus unix machines to test performance.
\end{question}
\begin{question}{Getting started}
We have not done much apart from writing the proposal so far. We do not foresee
any trouble with starting our project immediately.
\end{question}

\end{document}
