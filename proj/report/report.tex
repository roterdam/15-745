\documentclass[a4paper,twocolumn]{article}
\usepackage{times}
\usepackage[english]{babel}
\usepackage[utf8x]{inputenc}
\usepackage{amsmath}
\usepackage{graphicx}
\usepackage{latexsym}
\usepackage{pgfplots}
\usepackage{hyperref}
\usepackage{tikz}
\usepackage[colorinlistoftodos]{todonotes}
\usepackage[left=2cm,right=2cm,top=2cm,bottom=2.5cm]{geometry}
\usepackage{authblk}
\usepackage{syntax}
\title{Improving Cache Locality with Coalesced Sequence operations}
\date{\vspace{-5ex}}
\author{Rokhini Prabhu\thanks{rokhinip@gmail.com} }
\author{Michael Choquette\thanks{mchoquet@andrew.cmu.edu}}
\affil{Carnegie Mellon University}

\definecolor{bblue}{HTML}{4F81BD}
\definecolor{rred}{HTML}{C0504D}
\definecolor{ggreen}{HTML}{9BBB59}
\definecolor{ppurple}{HTML}{9F4C7C}


\newcommand{\sml}[1]{\texttt{#1}}

\begin{document}
\maketitle

\begin{abstract}
Immutable sequences are a common data structure in several functional languages; they
are usually operated on using constructs like \sml{map, filter} and
\sml{reduce}, and it's common for code to chain multiple such operations together. In this
project, we explore the possibility of an optimization which improves locality and
performance on this code pattern. We do so by statically analyzing the source code and
coalescing sequence operations together, thereby reducing the number
of passes through memory and improving locality. This optimization is a
markedly different approach compared to other cache optimizations such as tiling and
rearranging loops for we are delaying computations and sometimes,
not performing them at all. We have found that by performing this
optimization, on our 5 benchmark tests, we have obtained $<INSERT_NUMBER_HERE>$
fewer cache misses on average.
% TODO: Insert any last minute comment on how this is good/bad/not expected
\end{abstract}

\section{Approach}

\subsection{Motivation}
Consider the following code which does a simple map-reduce operation.

\begin{verbatim}
// Constructor function
int A_constr(int i);

// Construct sequence A
int<> A = tabulate A_constr 1000000
int<> A' = map f A
int res = reduce (+) 0 A'
\end{verbatim}

\noindent
Assuming that sequences are implemented as arrays, we observe the following:
Each location in the sequence is accessed thrice - during the function calls to
\sml{tabulate}, \sml{map} and \sml{reduce} functions. We can instead compose
\sml{A_constr} with \sml{f} and replace the code with the following:
\begin{verbatim}
int A = tabulate (A_constr . f) 1000000
int res = reduce (+) 0 A
\end{verbatim}

\noindent
From this small motivating example, we can see how combining operations can
reduce memory accesses, and potentially provide large performance improvements.

\subsection{Approach Overview}
C0 is a safe subset of the C programming language developed here at CMU by
Frank Pfenning. We chose to work with C0 to implement the above optimizations
since it is a small and lightweight language and also due to our familiarity with it
from 15-411. First we modified our starting compiler to generate C code instead
of X-86 and added sequences as a primitive datatype, and then we wrote a series
of transformation passes to coalesce sequence operations where possible. Our
compiler currently has the following pipeline:
\begin{enumerate}
  \item Parser
  \item Elaboration
  \item Typechecker
  \item Optimization Passes
  \item Translation from Abstract Syntax Tree to C code
\end{enumerate}
\setlength{\grammarparsep}{5pt plus 1pt minus 1pt} % increase separation between rules
\setlength{\grammarindent}{10em} % increase separation between LHS/RHS

\subsection{Language Design}

\newcommand{\nonterm}[1]{$\langle${#1}$\rangle$}
\newcommand{\OR}{\ensuremath{\ | \ \ }}
\newcommand{\term}[1]{\textbf{#1}}
\newcommand{\code}[1]{\texttt{#1}}

We are approaching this problem by writing a source-to-source compiler that
translates augmented C0 code to C code. Therefore, we have to set up some
groundwork that involves grammar changes as well as specifying type checking
rules for sequences. We have extended the C0 grammar that is defined in
\href{http://c0.typesafety.net/doc/c0-reference.pdf}{\textit{C0 language
    specification}} as specified in Figure 1.
The terminal \term{ident} is used to refer to an identifier.

\begin{tabular}{l}
  \\
\nonterm{exp-no-seq} ::= \code{(} \nonterm{exp} \code{)} \OR \nonterm{num}
\OR \nonterm{strlit} \OR ... \\ \\
\nonterm{sequence-follow} ::= $\epsilon$ \OR \code{,} \nonterm{exp-no-seq} \nonterm{sequence-follow} \\
\nonterm{sequence} ::= \code{< >} \OR \code{<} \nonterm{exp-no-seq}
														\nonterm{sequence-follow}
														\code{>} \\
			      \OR \code{<}\nonterm{exp}\code{..}\nonterm{exp}\code{>}
			  \OR \code{tabulate} \term{ident} \nonterm{sequence} \\
			      \OR \code{map} \term{ident} \nonterm{sequence} \\
			      \OR \code{reduce} \term{ident} \nonterm{exp-no-seq} \nonterm{sequence} \\
			      \OR \code{filter} \term{ident} \nonterm{sequence} \\\\
\nonterm{exp} ::= \nonterm{exp-no-seq} \OR \nonterm{sequence} \\\\
\end{tabular}
Figure 1.0: Extension of C0 grammar to support parallel vectors.

\subsection{Static Semantics}
The static semantics and type-checking rules are also affected as follows:
\\
$$\frac{ }{\cdot \vdash \code{<>} : seq\;\tau}$$.
\\
$$\frac{\forall\; x \in arglist, \Gamma \vdash x : \tau }{\Gamma \vdash \code{<}arglist\code{>}: seq\;\tau}$$.
\\
$$\frac{\Gamma \vdash f : int \to \tau, \Gamma \vdash n : int}
  {\Gamma \vdash \code{tabulate}\;\code{f}\;n : seq\;\tau}$$
\\
$$\frac{\Gamma \vdash f : \tau \to \tau', \Gamma \vdash s : seq\;\tau}
  {\Gamma \vdash \code{map}\;\code{f}\;s : seq\;\tau'}$$
\\
$$\frac{\Gamma \vdash f : (\tau * \tau) \to \tau, \Gamma \vdash s : seq\;\tau,
  \Gamma \vdash b : \tau}
  {\Gamma \vdash \code{reduce}\;\code{f}\;b\;s : \tau}$$
\\
$$\frac{\Gamma \vdash f : \tau \to bool, \Gamma \vdash s : seq\;\tau}
  {\Gamma \vdash \code{filter}\;\code{p}\;s : seq\;\tau}$$
\\
  While the rules above specify a general type $\tau$, we restrict $\tau$ to be
  to be either an $int$, $bool$ or $char$. This is because \code{map},
  \code{filter} and \code{reduce} operations on pointers and arrays are not
  reasonable or safe operations.
  \\\\
  We also enforce that sequences are immutable in our type-checking. This means that
  their elements cannot be written to, only read, which is in line with the use of
  sequences in a functional context and also allows for simpler optimization passes.
\subsection{Optimization Passes}

There are 3 sections in particular that we will explain here - performing
modified reverse lazy code motion on the sequence operations, combining sequence
operations together and finally translating to the target language. It is worth
noting that unlike the optimizations that we have discussed in class, this
optimization occurs on the Abstract Syntax Tree level. This gives us alot of
additional information - such as types - to work with but at the same time, it
is, far more complex than the usual 3 address form.
\\
\\
As a first pass, we perform reverse lazy code motion on sequence
operations. This means that \code{tabulate}, \code{map}, \code{reduce} and
\code{filter} operations are pushed as far down as they can possibly be within
a basic block. We do this because there might be code patterns as follows:
\begin{verbatim}
int<> A = tabulate A_constr 10000;
/* Large block of non-sequence
   related code */
int<> A' = map f A
\end{verbatim}
In such a case, we try to move the operations as close together as possible so
that we can identify opportunities for coalescing related sequence operations
that are close together.
\\
\\We are then faced with the question of how far one can push-down these
sequence operations. We take a more conservative approach to this compared to
LCM. We stop the code motion of an operation on sequence $S$ if the next
statement in program order, satisfies one of the following conditions:
\begin{enumerate}
\item Is a sequence operation - \code{map}, \code{filter}, \code{reduce}
on sequence $S$
\item Is a conditional - \code{if} statement
\item Is a loop
\end{enumerate}
The first condition is straightforward - it is due to a clear case of true
dependence. In the second case, we are conservative about moving
operations into conditionals since we don't know which branches will
be executed - we will have to ensure that all control flow paths eventually
perform the sequence operation which can lead to repeated code. To avoid this
scenario, we stop the code motion. In the third scenario, we avoid moving
sequence computations into loops for 2 reasons - loops could possibly never
execute which means we would need the computation to happen outside the loop
anyway and we would like to to avoid recomputing sequences
multiple times in the loop.
\\
\\As such, we perform this first pass as a backwards analysis - we identify
potential sequences during our backwards traversal. We then explore the
remainder of the upwardly exposed program to find the statement that defines
these sequences. This then allows us to perform a forward analysis to pull the
statements down as far down as possible. Note that as a result of moving a
sequence operation down, we might also have a cascading effect of being able to
move a related sequence down. Consider the following case:
\begin{verbatim}
int<> B = tabulate B_constr 10000;
int<> A = map f B;
/* Large block of non-sequence
   related code */
int res = reduce (+) 0 A
\end{verbatim}
In this case, as a result of performing code motion on the line \code{int<> A =
  map f B}, we will also be able to pull the line \code{int<> B = tabulate ...}
down as well. As such, we perform this optimization until convergence.\\

\noindent
Once all sequence operations have been postponed as far as possible, we have the guarantee that if two sequence operations can be directly combined, they're consecutive statements in the program. We use this to further simplify the code in anticipation of the third pass; all variable definitions and redefinitions that are used in exactly one place in the program are inlined, and the defining write line is deleted. We accomplish this by looping upwards through the code, where every time we encounter a sequence-producing operation we look at all the subsequent code to see if the variable is used before being redefined. In the case where inlining is allowed, we do so and remove the assignment line.\\

In our third pass, once all sequence operations that are used exactly once have been inlined, we look through all the expressions in the program and iteratively replace pairs of nested operations with single operations on combined functions, as in the example above. All the newly defined functions are added to the AST at the end of the pass. The types of replacements we support are listed below in pseudocode:
\begin{verbatim}
map(f2, tabulate(f1, n))
  =>  tabulate(f2 . f1, n)
map(f2, map(f1, s))
  =>  map(f2 . f1, s)
filter(f2, filter(f1, s))
  =>  filter(f2 && f1, s)
map(f2, combine(f, s1, s2))
  =>  combine(f2 . f, s1, s2)
\end{verbatim}


\noindent
Once all sequence operations have been combined, we convert to C code by doing the following:
\begin{enumerate}
\item Scan the code to make a set of all the functions passed as arguments to each map operation.
\item For each (operation, fnName), add a function that performs that operation on the input, calling the given function where appropriate.
\end{enumerate}
\noindent
Note that we do not have a single implementation of each of \sml{map}, \sml{reduce}, \sml{filter}, etc, because that would require passing a function pointer, and the overheads of calling function pointers so frequently would likely hide any speedups gained by our optimization. Instead, we generate a separate clone of the function for each input function it's passed to lower overheads and maximize the possibility of inlining by gcc.




\section{Experimental Setup}
The experimental setup for project is simple. We measure the performance gains
of our optimization two-fold. We first time the execution time of the final
binary and compare the final results. Secondly, for a more nuanced picture, we
also use \code{valgrind}, specially with the \code{cachegrind} option for
information about number of data cache hits and misses.
\\
\\In both of these measurement scenarios, we are influenced heavily by the
state when the code is run. Since it is not possible for us, without kernel
privileges, to be sure of the cache state, we attempt to achieve a clean cache
in between the runs of our program. We do this by running an IO-intensive
program which writes to about $5M$ of memory - which is much bigger than the
size of L1 and L2 caches put together. While there are caveats with this approach
since one can never be sure of the mapping of data to cache lines, it is the
best solution we could come up with in order to thrash the cache in between
runs of our benchmark tests.
\\
\\ The experiments mentioned above have been performed on a Intel
$<INSERT_SPECS_FOR_MACHINE_HERE>$.
\section{Evaluation}

\subsection{Tests}
% TODO: Include diagrams specifying speedups/slowdowns.
% Analyze performances - why it was faster or wasn't

\subsection{Lessons learnt}

\section{Conclusion and Future work}

\section{Appendix} % Optional, you can make this a biblio if you want


\end{document}
